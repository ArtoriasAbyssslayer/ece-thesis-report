\section{Σκοπός της διπλωματικής εργασίας}
\label{section:Contribution}

Η έρευνα που πραγματοποιείται, είναι μια από τις πρώτες εκτενείς προσπάθειες επίλυσης του προβλήματος της αξιόπιστης ανακατασκευής \enit{3D} επιφανειών, με την έννοια του φωτορεαλισμού. To πρόβλημα αντιμετωπίζεται, μέσα από εκπαίδευση έμμεσων αναπαραστάσεων και την χρήση λίγων εικόνων, διαφόρων όψεων, ως μέσο επίβλεψης στην διαδικασία αντίστροφης νευρωνικής αποτύπωσης.

Τα νευρωνικά δίκτυα εκ φύσεως αδυνατούν να συλλάβουν το υψηλοσυγχνοτικό περιεχόμενο της ακατέργαστης πληροφορίας, κάτι που πηγάζει από τον τρόπο εκπαίδευσης και λειτουργίας τους. Επομένως, είναι από τις λίγες εργασίες που δίνει κυρίαρχη έμφαση στην κωδικοποίηση των έμμεσων αναπαραστάσεων της εισόδου που δίνονται στο δίκτυο. Κύριος στόχος αυτού, η ανακατασκευή να έχει αξιόπιστα αποτελέσματα σε υψηλοσυχνοτικές περιοχές αλλά και ταχύτερη σύγκλιση. 

Στην επιστημονική κοινότητα δεν έχει ακόμα διακριθεί ποια είναι η ιδανική κωδικοποίηση περιεχομένου για την συγκεκριμένη εφαρμογή. Οι περισσότερες εργασίες αντιμετωπίζουν το πρόβλημα ως πρόβλημα υπερεκπαίδευσης πεδίων απόστασης. Σύγχρονες μελέτες δείχνουν όμως, πως υπάρχει χώρος επίλυσης, με συνδυασμό μεθόδων κωδικοποίησης.

Συνεπώς, στην παρούσα διπλωματική εργασία εξετάζεται η χρήση κωδικοποιήσεων και η σύλληψη υψηλοσυχνοτικού περιεχομένου για την εκπαίδευση νευρωνικών δικτύων που χρησιμεύουν ως εκτιμητές συνεχών συναρτήσεων(\enit{function approximators}) πάνω στο πρόβλημα \enit{3D} ανακατασκευής μοντέλων, φωτογραφημένων αντικειμένων. Αυτή η εκτενής χρήση δικτύων κωδικοποίησης μετασχηματίζει την είσοδο και οδηγεί το δίκτυο να συγκλίνει πιο γρήγορα σε ακριβείς εκτιμήσεις ανεξαρτήτως γεωμετρικής φύσης του 3D μοντέλου που ανακατασκευάζεται. Ταυτόχρονα φαίνεται να γενικεύει καλύτερα το δίκτυο ενώ αναπαριστά καλύτερα μεγάλες σκηνές.

