\section{Διατύπωση του Προβλήματος 3D Ανακατασκευής}
\label{section:problem_description}

\par
    Η ανακατασκευή τρισδιάστατων αναπαραστάσεων αναφέρεται σε όλες τις μεθόδους που έχουν ως στόχο να ανακτήσουν την πληροφορία της γεωμετρίας και της εμφάνισης ενός \enit{3D} πραγματικού αντικειμένου.
\par
    Το πρόβλημα της \enit{3D} ανακατασκευής λόγω των πολλών εφαρμογών της, έχει ερευνηθεί εκτενώς με διάφορες προσεγγίσεις. Σε αυτές τις μεθόδους συμμετέχουν οι ενεργές μέθοδοι που αναφέρονται στην αριθμητική αντιμετώπιση του προβλήματος. Αυτό επιχειρείται, μέσα από έναν χάρτη βάθους \enit{(depth map)} ο οποίος αποκτάται χρησιμοποιώντας μεθόδους όπως το <<δομημένο φως>> \enit{(structured light)}. Ενδεικτικές εφαρμογές, οι χρονοβόρες τρισδιάστατες σαρώσεις  \enit{3D scans}.
\par
    Από την άλλη πλευρά, υπάρχουν οι παθητικές μέθοδοι που δεν παρεμβαίνουν άμεσα στο ανακατασκευασμένο αντικείμενο. Αντίθετα, χρησιμοποιούν πληροφορίες μόνο από την ακτινοβολία που ανακλάται από τα αντικείμενα και αποτυπώνεται στον αισθητήρα της κάμερας. Συνεπώς, τα δεδομένα που χρησιμοποιούν είναι μόνο οι πληροφορίες που παρέχει μια φωτογραφία. Σε αυτόν το τομέα, υπάρχουν κλασσικές μέθοδοι που εξετάζουν την απόκτηση \enit{3D} σχήματος από την σκίαση που εφαρμόζεται \enit{Shape-from-Shading}, άλλες από την υφή που παραμορφώνεται κατά την προβολή \enit{Shape-from-texture}, και μία άλλη πιο εκλεπτυσμένη προσέγγιση που αναφέρεται στην φωτομετρική στερεογραφία ή φωτογραμμετρία. Η τελευταία ανακτά την πληροφορία βάθους από φωτογραφίες του αντικειμένου σε διαφορετικές συνθήκες φωτισμού. Σε σύγκριση με τις ενεργές μεθόδους αυτές οι μέθοδοι έχουν πολύ περισσότερα πεδία εφαρμογής.
\par 
    Όλες οι προηγούμενες μέθοδοι αντιστοιχούν σε πρακτικές που χρησιμοποιούνται σήμερα στους τομείς της υγείας και της βιομηχανίας, αλλά είναι εξαιρετικά αργές και δεν μπορούν να χρησιμοποιηθούν σε πραγματικό χρόνο. Για αυτόν τον λόγο, η στερεοπτική με υπολογιστές  (\enit{Computer Stereo Vision}) έχει αναπτύξει αλγορίθμους που χρησιμοποιούν ταυτόχρονα εικόνες από διάφορες θέσεις καμερών αντιστρέφοντας το πρόβλημα. Συγκεκριμένα προσπαθούν να υπολογίσουν τις τρισδιάστατες συντεταγμένες του αντικειμένου κάνοντας τεχνικές αντίστροφες από την προβολή που κάνει η κάμερα. Για να γίνει αυτό, χρησιμοποιούνται μέθοδοι  αποτύπωσης γραφικών σε εικόνα που λαμβάνουν υπόψιν τον ολικό φωτισμό της σκηνής με μεθόδους όπως η ιχνηλάτιση ακτίνων των οποίων η ρίψη ξεκινά από την κάμερα και αλγόριθμοι που εκτιμούν την τομή των ακτίνων που περνούν από την εικόνα πάνω στο \enit{3D} μοντέλο. Αυτές οι μέθοδοι, απαιτούν τον υπολογισμό των σημείων τομής ακτίνας επιφάνειας, τον υπολογισμό του ανακλώμενου χρώματος και ταυτόχρονα η αποτυπωμένη εικόνα πρέπει να μοιάζει με την πραγματική φωτογραφία.
 
\section{Διατύπωση προβλήματος 3D ανακατασκευής επιφανειών με έμμεση μορφή ως πρόβλημα βελτιστοποίησης με νευρωνικά δίκτυα}
\par
    Στο πεδίο της Υπολογιστικής Όρασης (\enit{Computer Vision}) και των \enit{3D} Γραφικών οι επιφάνειες συνήθως αναπαριστώνται με άμεσους τρόπους.Τέτοιες αναπαραστάσεις είναι είτε πολυγωνικές(συνήθως τρίγωνα με σημεία στον \enit{3D} χώρο για κορυφές), είτε ογκομετρικές αναπαραστάσεις (νέφη σημείων \enit{Point Clouds}).
\par
    Μετά από διάφορα εγχειρήματα βελτίωσης του προβλήματος της ανακατασκευής, η επιστημονική κοινότητα κατέληξε πως οι άμεσες αναπαραστάσεις όπως ένα ακατέργαστο νέφος σημείων αυξάνουν την πολυπλοκότητα των αλγορίθμων. 
\par
    Σε αυτό το πλαίσιο προτάθηκαν οι έμμεσες αναπαραστάσεις γραφικών οι οποίες αλλάζουν το πρόβλημα από την ανακατασκευή μια τρισδιάστατης επιφάνειας, στον υπολογισμό ενός πεδίου προσημασμένης απόστασης \enit{SDF}. Το πεδίο αυτό, αντιστοιχεί κάθε σημείο τους χώρου σε μια προσημασμένη απόσταση από την επιθυμητή προς ανακατασκευή επιφάνεια. Στην συνέχεια με έναν αλγόριθμο ισοδυναμικής επιφάνειας εξάγεται η άμεση αναπαράσταση από το πεδίο(βλ.Κ\ref{section:appendix-algorithms} \enit{Marching Cubes} ) 
\subsection*{Πώς συνδέεται η τεχνητή νοημοσύνη;}
\par
    Με την ανάπτυξη των τεχνικών εκπαίδευσης των τεχνητών νευρωνικών δικτύων (\enit{ANNs}) και την απαγκίστρωση μας από το γενικό θεώρημα προσέγγισης (\enit{General Approximation Theorem}), είμαστε πλέον στην άνθιση των βαθιών νευρωνικών δικτύων(\enit{DNNs}). Συνεπώς το πρόβλημα εκτίμησης των έμμεσων αναπαραστάσεων γραφικών μπορεί να λυθεί με νευρωνικά δίκτυα.
\par
    Οι μέθοδοι που προτάθηκαν δίνουν εκλεπτυσμένες μεθόδους για την αποτύπωση των γραφικών σε εικόνα με διαφορίσιμη μάλιστα μορφή, μετατρέποντάς το σε πρόβλημα βελτιστοποίησης που μπορούν να το χειριστούν τα βαθιά νευρωνικά δίκτυα. Παρ' όλα αυτά τα δίκτυα συντεταγμένων, που αναπαριστούν \enit{SDF}, μέχρι σήμερα δεν είναι δίκτυα που εκπαιδεύονται γρήγορα και παρουσιάζουν προβλήματα σε περιοχές με υψηλοσυχνοτικές μεταβολές της γεωμετρίας. 
\par    
    Αυτό το πρόβλημα αποσκοπεί να ερευνήσει η παρούσα εργασία προτείνοντας μεθόδους που βασίζονται στην υψηλοσυχνοτική κωδικοποίηση εισόδου.


