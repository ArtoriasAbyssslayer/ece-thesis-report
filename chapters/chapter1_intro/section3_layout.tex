\section{Δομή Διπλωματικής Εργασίας}
\label{section:Layout}
\par 
    Η διπλωματική εργασία ασχολείται με δύο αντικείμενα. Αρχικά το κομμάτι του δικτύου που είναι υπεύθυνο για την εκτίμηση της γεωμετρίας της επιφάνειας στον τρισδιάστατο χώρο μέσω πεδίων προσημασμένης απόστασης (\enit{SDF}). Αυτό, πραγματοποιείται μέσα  από την έμμεση μορφή αντίστροφης αποτύπωσης διαφορίσιμων ισομετρικών επιφανειών (\enit{Implicit Differentiable Rendering}), με ξεχωριστό τον υπολογισμό του πεδίου ακτινοβολίας (\enit{Rendering Network}). 
\par 
    Δεύτερο και πεδίο έρευνας της παρούσας εργασίας, είναι οι αλγόριθμοι και τα δίκτυα βαθιάς
    κωδικοποίησης των έμμεσων πεδίων αναπαράστασης γεωμετρίας και φωτισμού.
\par    
    Επομένως, πρώτα γίνεται μια ανάλυση του θεωρητικού υπόβαθρου πάνω στις γραφικές αναπαραστάσεις στους υπολογιστές και στις μεθόδους που χρησιμοποιούνται για την την αποτύπωση και κωδικοποίηση αυτών. Στην συνέχεια, παρουσιάζονται οι πιο πρόσφατες προσπάθειες από την επιστημονική κοινότητα, στην προσέγγιση του προβλήματος με αντίστοιχες ερευνητικές εργασίες. Εργασίες, σε οποίες βασίζεται και η παρούσα εργασία, χρησιμοποιώντας παράλληλα και μοντέλα για την απόδοση και του χρώματος των σκηνών ξεχωριστά από την γεωμετρία. Στην συνέχεια αναλύεται το θεωρητικό υπόβαθρο που απαιτείται, δίνεται η υλοποίηση που έγινε πάνω στο δίκτυο κωδικοποίησης της εισόδου των δικτύων ανακατασκευής επιφάνειας (το οποίο μπορεί να εφαρμοστεί και σε δίκτυα απόδοσης όγκου \cite{mildenhall2020nerf}, \cite{DBLP:journals/corr/abs-2106-10689}), αλλά και οι μετατροπές στα ίδια τα δίκτυα εκτίμησης της ισομετρικής επιφάνειας του \enit{3D} μοντέλου. Συνεπώς η εργασία διαρθρώνεται ως εξής:
\small{
\begin{itemize}
  \item{\textbf{Κεφάλαιο \ref{chapter:theory}:} Περιγράφονται  θεωρητικά στοιχεία στα οποία βασίστηκαν οι υλοποιήσεις. Γίνεται εισαγωγή στην επιστήμη της γραφικής υπολογιστών, της υπολογιστικής όρασης καθώς και της βαθιάς μάθησης και στην συνέχεια δίνεται βάση, στο πώς αυτές οι επιστήμες συνδυάζονται στο κομμάτι της ανακατασκευής μοντέλων. Επιπλέον, παρουσιάζεται το θεωρητικό υπόβαθρο αλγορίθμων αποτύπωσης \enit{3D} σκηνής σε εικόνα από έμμεσες επιφάνειες(\enit{implict differentiable rendering} \cite{sitzmann2020scene, yariv2020multiview, chen2023review}), ενσωμάτωσης πληροφορίας και εκτίμησης σημείων μέσω πολυ-στρωματικών νευρωνικών δικτύων.
    }
  \item{\textbf{Κεφάλαιο \ref{chapter:sota}}
      Γίνεται ανασκόπηση της ερευνητικής περιοχής που αφορά την ανακατασκευή \enit{3D} σκηνής, αλλά και των τεχνικών με τις οποίες δίνεται δυνατότητα στα δίκτυα εκτίμησης \enit{3D} σκηνής να μάθουν το υψηλοσυχνοτικό περιεχόμενο των δεδομένων.}
  \item{\textbf{Κεφάλαιο \ref{chapter:implementations}:} Πλήρης περιγραφή των υλοποιήσεων διαφόρων μοντέλων \enit{(Implicit Differential Network \cite{yariv2020multiview}, Rendering Network \cite{mildenhall2020nerf}, Multi-Resolution Hash Embedding \cite{mueller2022instant}, NFFB\cite{wu2023neural}, Style Modulation \cite{vaswani2023attention, huang2017arbitrary}} με βάση τις τεχνικές που ακολουθήθηκαν για την βελτιστοποίηση νευρωνικών ανακατασκευής \enit{3D} σκηνών.} 
  \item{\textbf{Κεφάλαιο \ref{chapter:experiments}:} Παρουσιάζεται  η μεθοδολογία των
      πειραμάτων και τα αποτελέσματα.
    }
  \item{\textbf{Κεφάλαιο \ref{chapter:conclusions}:} Παρουσιάζονται τα τελικά συμπεράσματα, ενώ ταυτόχρονα αναφέρονται τα
      προβλήματα που προέκυψαν και προτείνονται θέματα για μελλοντική
      μελέτη, αλλαγές και επεκτάσεις.
    }
 \item{\textbf{Παράρτημα κ.\ref{chapter:appendix}:} Αφήνονται για τον αναγνώστη πολλές σημαντικές πληροφορίες που δεν μπορούσαν να ενταχθούν άμεσα στο βασικό κείμενο της εργασίας}
\end{itemize}
