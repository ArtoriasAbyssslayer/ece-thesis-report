\section{Διαμόρφωση/Αποδιαμόρφωση Χαρακτηριστικών Σήματος}
\par
    H τεχνική που ακολουθείται στην διαμόρφωση και αποδιαμόρφωσης βαρών βασίζεται στον αλγόριθμο \enit{AdaIN(Adaptive Instance Normalization}, ο οποίος δεν λειτουργεί σε ομάδα δεδομένων αλλά σε κάθε δεδομένο χαρακτηριστικού ξεχωριστά. 

\par    
    Η τεχνική αυτή αποδίδεται με χρήση ενός \enit{Style Modulation Block} το οποίο εισάγεται στα γραμμικά στρώματα της νευρωνικής θυρίδας φίλτρων Fourier (\enit{NFFB}) και χρησιμοποιεί την είσοδο ως δεδομένο και την κωδικοποίηση Fourier σε δεδομένη ανάλυση του πλέγματος κατακερματισμού ως διάνυσμα χαρακτηριστικών. Αντί να εφαρμόσει τον κλασσικό αλγόριθμο προσαρμοστικής κανονικοποίησης πάνω στις στατιστικές του διανύσματος χαρακτηριστικών που περιγράφεται με την εξίσωση:
    \begin{equation}
    \text{AdaIN}(x, y) = \mu(y) + \sigma(y) \cdot (\frac{x - \mu(x)}{\sigma(x)}),        
    \label{eq:AdaIN}
    \end{equation}

    εφαρμόζεται ένα επίπεδο προσοχής το οποίο υπολογίζει τα βάρη προσοχής ως το εσωτερικό γινόμενο πραγματικής εισόδου με διάνυσμα χαρακτηριστικών. Αυτά τα βάρη κανονικοποιούνται με συνάρτηση \enit{Softmax} στο διάστημα (0, 1) και χρησιμοποιούνται για τον υπολογισμό του αποδιαμορφωμένου διανύσματος χαρακτηριστικών. Έτσι εφαρμόζεται η τεχνική διαμόρφωσης και αποδιαμόρφωσης των βαρών του δικτύου χωρίς να βασιζόμαστε σε στατιστικές των δεδομένων ενός συνόλου εισόδων.

    