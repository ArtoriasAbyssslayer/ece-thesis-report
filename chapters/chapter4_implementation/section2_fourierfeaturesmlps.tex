\section{Συχνοτική Κωδικοποίηση Εφαπτόμενου Πυρήνα Συναρτήσεων Fourier}
\label{section4:FourierFeatureNets}
Η εργασία βασιζόμενη στην ανάλυση που έγινε στο θεωρητικό υπόβαθρο (βλ. \ref{section:dnnEmbeddings}), προτείνει την βελτιστοποίηση της σύγκλισης του σφάλματος κόστους των νευρωνικών δικτύων μέσω παραμετροποιήσιμων στάσιμων εφαπτόμενων κωδικοποιήσεων στην είσοδο. Καταλήγει, έτσι σε τρεις τύπους εφαπτόμενων πυρήνων που είναι σε θέση να αποτυπώσουν υψηλοσυχνοτικές σχέσεις των δεδομένων, οι οποίες είναι καλό να εφαρμόζονται σε δίκτυα συντεταγμένων (κρίσιμο όταν πρόκειται για δεδομένα εικόνας από πολλές όψεις). Αυτοί οι εφαπτόμενοι πυρήνες \enit{Neural Tangent Kernels} διαφέρουν ως προς την μορφή των συναρτήσεων πυρήνα που χρησιμοποιούνται και είναι οι εξής:

\begin{itemize}
    \item Βασική κωδικοποίηση απεικονίσεων όπου απλά γίνεται μετασχηματισμός των εισόδων γύρω από κύκλο ή σφαίρα με χρήση ημιτονοειδών συναρτήσεων με NTK:
        \[ \gamma(\mathbf{v})=\left[\cos(2\pi\mathbf{v}), \sin(2\pi\mathbf{v})\right]^{\mathrm{T}}\]
    \item Κωδικοποίηση με βάση την θέση των συντεταγμένων - \enit{Positional Encoding}. Γίνεται δειγματοληψία συχνοτήτων πάνω σε λογαριθμική κλίμακα σε κάθε διάσταση των συντεταγμένων που κωδικοποιούνται και το εύρος $\sigma$ επηρεάζει την δυνατότητα του μοντέλου να γενικεύει ή και την ακρίβειά του (είναι παράμετρος που εκπαιδεύεται, και τα αποτελέσματά της είναι εμφανή στα σχήματα \ref{fig:wideNTK} \ref{fig:narrowNTK})). Το εφαπτόμενο πεδίο συναρτήσεων πυρήνα περιγράφεται από την παρακάτω μαθηματική μορφή:
    \[
    \gamma(\mathbf{v})=\left[\ldots, \cos \left(2 \pi \sigma^{j / m} \mathbf{v}\right), \sin \left(2 \pi \sigma^{j / m} \mathbf{v}\right), \ldots\right]^{\mathrm{T}} \text { για } j=0, \ldots, m-1
    \]
    \item Κωδικοποίηση με τυχαία Γκαουσιανή κατανομή. Κρίθηκε πως η κατανομή δειγματοληψίας των συχνοτήτων και οι ίδιες οι συχνότητες $\mathbf{B}$,  είναι και αυτές παράμετροι προς εκπαίδευση και μπορούν να δειγματοληπτηθούν αρχικά από κάποια κανονική κατανομή $\mathcal{N}(0, \sigma^2)$ 
    \[
    \gamma(\mathbf{v})=[\cos (2 \pi \mathbf{B} \mathbf{v}), \sin (2 \pi \mathbf{B} \mathbf{v})]^{T}
    \]
\end{itemize}
\par 
    Έτσι, γίνεται χρήση ενός πυρήνα ημιτονοειδών συναρτήσεων κωδικοποίησης στο IDR ή μετασχηματιστή (\enit{transformer}), θα μπορούσαμε να πούμε, $\gamma(x;\theta, \gamma, \tau)$ το οποίο στην μία εκδοχή του χαρακτηρίζεται Fourier Features όταν χρησιμοποιεί ημιτονοειδή συναρτήσεις πυρήνα σε συχνότητες από Gaussian κατανομή.

    Σε άλλη εκδοχή του πυρήνα συναρτήσεων κωδικοποίησης, έχουμε \enit{positional embedding} μετασχηματισμό εισόδου, που η δειγματοληψία σχετίζεται με την θέση. Ωστόσο και τα δύο δείχνουν ότι η εκπαίδευση τους δίνει παρόμοια αποτελέσματα χωρίς ιδιαίτερη επιτάχυνση της σύγκλισης στην εκπαίδευση του \enit{IDR}.
    \clearpage