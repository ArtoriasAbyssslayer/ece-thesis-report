\section{Συμβολή της Εργασίας - Συμπεράσματα}
\label{contribution}

Η παρούσα εργασία καταφέρνει να εξετάσει μια πληθώρα μεθόδων κωδικοποίησης σημείων και διανυσμάτων όψης, με σκοπό την αξιόπιστη τρισδιάστατη ανακατασκευή. Έτσι τα δίκτυα αυτά μπορούν να αποτελέσουν μια λύση όσον αφορά το πεδίο των \enit{Vision Transformers} και συγκεκριμένα των \enit{3D Vision coordinate based transformers}. 

Φυσικά η έρευνα χρειάζεται επέκταση και μετατροπή σε κώδικα χαμηλότερου επιπέδου με σκοπό την αποφυγή προβλημάτων διαρροής μνήμης που υπάρχουν. Τα αποτελέσματα ωστόσο εξακολουθούν να ξεπερνούν σε επίπεδο φωτογραφίας τα αποτελέσματα του IDR ειδικά σε λεπτομέρειες όπως για παράδειγμα στην αποτύπωσης της υφής του χρυσού κούνελου(σκηνή 110), η οποία αποτελεί από τις πιο δύσκολες σκηνές όσον αφορά την \enit{3D} ανακατασκευή.

Σαφώς δεν έχουν εξεταστεί όλες οι περιπτώσεις δεδομένων και ίσως θα έπρεπε το δίκτυο να μην εκπαιδεύεται σε τόσο μεγάλη κλίμακα αφού βλέπουμε στα τελευταία μοντέλα \enit{StylemodNFFB, StylemodNFFB\_TCNN} πως ήδη από τις πρώτες 100 εποχές έχει αποτυπωθεί λεπτομέρεια.

Τέλος παρατηρούνται \enit{artifacts} σε περιπτώσεις που κωδικοποιείται σύνθετο περιεχόμενο φωτισμού με απλό περιεχόμενο φωτισμού στα πλήρως συγχωνευμένα δίκτυα (\enit{TCNN}) του \enit{StylemodNFFB} για μεγάλη παράμετρο μεγέθους πίνακα κατακερματισμού και επιπέδων κατακερματισμού. Ίσως θα μπορούσε το ίδιο το δίκτυο \enit{NFFB} να αποτελέσει το έμμεσο πεδίο διαφορίσιμης γεωμετρίας ώστε να μην κάνουν μεγάλη οπισθοδιάδοση τα σφάλματα και υπάρχει θέμα τις κλίσεις σφαλμάτων όπως το σφάλμα μάσκας που είναι ήδη μικρό. \footnote{Σε αυτές τις περιπτώσεις παρατηρείται να έχει μπερδευτεί το δίκτυο γεωμετρίας με το δίκτυο εμφάνισης και ενώ το χρώμα είναι καλό και το σφάλμα μάσκας μικρό, να έχουμε μέρη γεωμετρίας που δεν είναι πάνω στο μοντέλο}