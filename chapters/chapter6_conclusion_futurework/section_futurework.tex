\section{Μελλοντικές Βελτιώσεις/Επεκτάσεις}
\label{section:futureworks}
Η προσπάθεια κωδικοποίησης υψηλοσυχνοτικού περιεχομένου με χρήση τρισδιάστατων δικτύων κωδικοποίησης για το μέρος της έρευνας που αφορά την ανακατασκευή τρισδιάστατων επιφανειών μέσω έμμεσων νευρωνικών αναπαραστάσεων, παρείχε σαφή βελτίωση τόσο στην ταχύτητα εκπαίδευσης των πεδίων προσημασμένης απόστασης αλλά και στην αξιοπιστία της ανακατασκευής υψηλοσυχνοτικών περιοχών. 

Βέβαια η πραγματική δύναμη του \enit{Multi-Resolution HashGrid Encoder} και κατά συνέπεια και του \enit{NFFB, StylemodNFFB} θα μπορούσε να φανεί στην ανακατασκευή επιφανειών μεγάλων εκτάσεων όπου η λεπτομέρειες είναι μικρής κλίμακας και πολύ περισσότερες. Δυστυχώς, δεν υπήρχαν επαρκή δεδομένα για να γίνει τέτοια έρευνα μια και το δίκτυο απαιτεί ως είσοδο έστω παραμέτρους κάμερας έστω και εκτιμήσεις αυτών. 

Δεδομένου του παραπάνω μια μελλοντική επέκταση θα ήταν η εφαρμογή του τρισδιάστατου κωδικοποιητή συνταγμένων σε ήδη υπάρχοντα δίκτυα ογκομετρικής απόδοσης όπως το \enit{NeRF}\cite{mildenhall2020nerf} και το \enit{NeuS} \cite{DBLP:journals/corr/abs-2106-10689} και παράλληλα αφαίρεση της επίβλεψης μέσω μασκών εικόνων κατά την ανακατασκευή που το παρόν δίκτυο χρειάζεται για να συγκλίνει ομαλά. Η διαδικασία επεκτείνεται με εφαρμογή των κωδικοποιήσεων σε εργασίες που κάνουν χρήση μονοσκοπικού βίντεο για συλλογή δεδομένων στα οποία εκπαιδεύεται το δίκτυο σε λίγα λεπτά. 

Τέλος μιας και η τάση της εποχής είναι τα παραμετρικά νευρωνικά πεδία (\enit{Conditional Neural Fields}) θα μπορούσε να γίνει εφαρμογή των μεθόδων που αναφέρθηκαν σε δίκτυα τα οποία χρησιμοποιούν έμμεσες αναπαραστάσεις ως \enit{latent code} (διάνυσμα εξόδου) σε παραγωγικά αντιπαραθετικά δίκτυα (\enit{GANs}) τα οποία θα μπορούν να παράγουν τυχαία μορφή \enit{3D} μοντέλων με τυχαίο χρώμα κάτι το οποίο θα οδηγούσε πιθανόν και σε βελτιωμένες μορφές απόδοσης παραμορφώσεων σε μηχανές γραφικών.
