\addcontentsline{toc}{section}{Περίληψη}
\begin{center}
\centering
  Σύνοψη \\  
  \vspace{1cm}
  \textbf{Ανακατασκευή 3D Επιφανειών μέσω έμμεσων αναπαραστάσεων με χρήση Δικτύων Κωδικοποίησης}
\end{center}
    \par 
    Το τελευταίο διάστημα υπάρχει άνθιση των επιτυχιών της βαθιάς μάθησης στην επεξεργασία εικόνων, καθώς και ανάγκη διαχείρισης \enit{3D} πληροφορίας σε εφαρμογές υπολογιστικής όρασης αλλά και σε μηχανές γραφικών. Μέχρι σήμερα επιστήμονες υγείας που ασχολούνται με τρισδιάστατα μοντέλα οστών/κοιλοτήτων, κατασκευαστές μηχανολογικών εξαρτημάτων, καλλιτέχνες, σχεδιαστές παιχνιδιών ή τεχνικοί οπτικών συστημάτων εφέ ταινιών που θέλουν να αποτυπώσουν εύκολα πραγματικά αντικείμενα στο υπολογιστή στις περισσότερες περιπτώσεις χρησιμοποιούν κλασσικούς τρόπους φωτογραμμετρίας για να δημιουργήσουν 3D σαρώσεις \enit{(3D scans)} αντικειμένων, μια διαδικασία που εκτός από χρονοβόρα είναι και άπληστη ως προς το πώς αναπαριστάται στη μνήμη το τελικό αποτέλεσμα. Σήμερα τα υψηλής ακρίβειας τρισδιάστατα μοντέλα είναι απαραίτητο να αποθηκεύονται στον υπολογιστή με λιγότερους δυνατούς χωρικούς συμβιβασμούς αλλά και η διαδικασία παραγωγής τους να είναι πιο διαισθητική και απλή παίρνοντας λίγες φωτογραφίες τους.
\par
    Στην παρούσα διπλωματική εργασία γίνεται μια προσπάθεια βελτίωσης των μεθόδων \enit{3D} ανακατασκευής μέσω αντίστροφης αποτύπωσης με χρήση νευρωνικών δικτύων. Αυτά τα βαθιά νευρωνικά δίκτυα, προσπαθούν να προσεγγίζουν ένα έμμεσο πεδίο γεωμετρίας και εμφάνισης μιας συμπαγούς επιφάνειας μέσω συνάρτησης προσημασμένης απόστασης, \enit{SDF (signed distance function)}. Τα δίκτυα αυτά δεν είναι σε θέση να συλλάβουν το περιεχόμενο χωρικής συχνότητας της επιφάνειας που αποτυπώνεται σε πολλαπλές προβολές (εικόνες που χρησιμοποιούνται για επίβλεψη). Συνεπώς, προτείνονται μέθοδοι που μετασχηματίζουν τον τρόπο λειτουργίας αυτών των δικτύων επιτρέποντας την αποτύπωση λεπτομερειών αλλά και τη σύλληψη υψηλοσυχνοτικών μεταβολών σε επίπεδο γεωμετρίας και εμφάνισης των επιφανειών.
    
\par
    Προτείνεται μια αρχιτεκτονική βαθιάς μάθησης, που βασίζεται σε προηγμένες τεχνικές επεξεργασίας σήματος. Για παράδειγμα, χρησιμοποιείται η ιδέα του μετασχηματισμού κυματιδίων (Wavelet Decomposition) μέσω Νευρωνικών Δικτύων αποθήκευσης συχνοτικού περιεχομένου (\textit{Neural Filter Banks} \cite{wu2023neural}) ώστε να αποδοθεί η καλύτερη δυνατή είσοδος για το νευρωνικό  δίκτυο. Σε αυτή την προσπάθεια υιοθετούνται καινοτόμες ιδέες σχετικά με την κωδικοποίηση όπως αυτή που προτείνει το \enit{Neural Graphics Primitives}\cite{mueller2022instant}, καταδεικνύοντας το πώς μπορεί μια κωδικοποίηση κατακερματισμού του χώρου σε πολλά επίπεδα ανάλυσης (\enit{Multi-Resolution hash encoding}) να αποτελέσει χωρική κωδικοποίησή του. Έτσι αυτό επιτρέπει στο δίκτυο να συλλάβει μεταβολές της γεωμετρίας και της εμφάνισης κατά τη σύνθεση μεγάλων σκηνών.
    Τέλος, γίνεται περαιτέρω επέκταση δίνοντας δυνατότητα προσοχής στο ποιοι χάρτες συχνοτικών χαρακτηριστικών των όψεων είναι πιο αποτελεσματικοί στην καλύτερη ανακατασκευή. Αυτό πραγματοποιείται κάνοντας διαμόρφωση και αποδιαμόρφωση των χαρτών χαρακτηριστικών της εισόδου των δικτύων έμμεσης αναπαράστασης (άτλαντας χαρακτηριστικών απεικόνισης). 
    