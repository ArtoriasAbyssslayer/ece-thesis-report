{\centering\LARGE Ακρωνύμια/Ορολογίες Εγγράφου\par}
\vspace{1em}
\label{append:acronyms}
\addcontentsline{toc}{section}{Ακρωνύμια/Ορολογίες}
\thispagestyle{empty}
Παρακάτω παρατίθενται ορισμένα από τα πιο συχνά χρησιμοποιούμενα ακρωνύμια και ορολογίες της παρούσας διπλωματικής εργασίας\footnote{Οι αγγλικοί όροι μεταφράζονται στα σημεία που χρησιμοποιούνται}:
\begin{table}[htpb]
  \centering
  \begin{tabularx}{\textwidth}{l@{$\;\;\longrightarrow\;\;$}X}
    3D/3Δ & 3 Dimensions | 3 Διαστάσεις\\
    2D/2Δ & 2 Dimensions | 2 Διαστάσεις \\
    Voxel & Volumetric Pixel | Ογκομετρικό Εικονοστοιχείο \\
    HF & High Frequency || High Fidelity \\
    \addlinespace
    CAGD & Computer Aided Geometric Design \\
    Ray Tracing & Ιχνηλάτιση Ακτινών \\
    Ray Marching & Αλγόριθμος Βάδισης πάνω στην ακτίνα \\
    Sphere Tracing & Σφαιρικός έλεγχος αλγορίθμου βάδισης \\
    BRDF & Bidirectional Reflectance Distribution Function \\
    \addlinespace
    NN & Neural Network \\
    ANN & Artificial Neural Network \\
    DL & Deep Learning \\
    DNN & Deep Neural Network \\
    MLP & Multilayer Perceptron \\
    CNN & Convolutional Neural Network \\
    RNN & Recurrent Neural Network \\
    ResNet & Residual Neural Network \\
    \addlinespace
    GAN & Generative Adversarial Network \\
    LSTM & Long Short-Term Memory Network \\
    CPU & Central Processing Unit \\
    GPU & Graphics Processing Unit \\
    \addlinespace
    SDF & Signed Distance Field \\
    IDR & Implicit Differentiable Rendering \\
    NEUS & Neural Surface Reconstruction \\
    NeRF & Neural Radiance Fields \\
    \addlinespace
    NFFB & Neural Fourier Filter Banks \\
    XOR & Exclusive OR \\
    MOD & Modulation \\
    StyleMod & Style Modulation \\
    AdaIN & Adaptive Instance Normalization \\
  \end{tabularx}
\end{table}
